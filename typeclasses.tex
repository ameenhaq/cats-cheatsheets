\documentclass{tufte-handout}

%\geometry{showframe}% for debugging purposes -- displays the margins

\usepackage{amsmath}

% Set up the images/graphics package
\usepackage{graphicx}
\setkeys{Gin}{width=\linewidth,totalheight=\textheight,keepaspectratio}
\graphicspath{{graphics/}}

% The following package makes prettier tables.  We're all about the bling!
\usepackage{booktabs}

% The units package provides nice, non-stacked fractions and better spacing
% for units.
\usepackage{units}

% The fancyvrb package lets us customize the formatting of verbatim
% environments.  We use a slightly smaller font.
\usepackage{fancyvrb}
\fvset{fontsize=\normalsize}

% Small sections of multiple columns
\usepackage{multicol}
\usepackage{multirow}

\newcommand{\fa}{F[A]}
\newcommand{\fb}{F[B]}
\newcommand{\rarr}{\texttt{=>}}
\newcommand{\fTwo}[2]{\texttt{#1} & \rarr & & & \texttt{#2}}
\newcommand{\fThree}[3]{\texttt{#1} & \rarr & \texttt{#2} & \rarr & \texttt{#3}}
\newcommand{\sdocUrl}[1]{https://typelevel.org/cats/api/cats/#1.html}
\newcommand{\sdocHref}[1]{\href{\sdocUrl{#1}}{#1}}

\title{cats Typeclass Cheat Sheet}
\author[Adam Rosien]{Adam Rosien (\href{mailto:adam@rosien.net}{adam@rosien.net})}
% if the \date{} command is left out, the current date will be used
% \date{2 December 2014}

\begin{document}

\maketitle% this prints the handout title, author, and date

\section{Installation}\label{sec:installation}

\noindent In your \texttt{build.sbt} file:

% \begin{fullwidth}
\begin{quote}
  \ttfamily libraryDependencies += "org.typelevel" \%\% "cats-core" \% "1.0.0-MF"
\end{quote}
% \end{fullwidth}

\noindent Then in your \texttt{.scala} files:

\begin{quote}
  \ttfamily import cats.\_
\end{quote}

\section{Defining Signatures}

Each typeclass is defined by a particular function signature and a set of laws\footnotemark (invariants) that the typeclass must obey.

\footnotetext{Typeclass laws are not listed here. See each typeclass' scaladoc link for more information.}

\begin{table}[ht]
  \centering
  \fontfamily{ppl}\selectfont
  \setlength{\tabcolsep}{5pt}
  \begin{tabular}{rcrcccl}
    Typeclass & \multicolumn{5}{c}{Signature} \\
    \midrule
    \sdocHref{Functor}               & \fThree{\fa}{(A => B)}{\fb} \\
    \sdocHref{Contravariant}         & \fThree{\fa}{(B => A)}{\fb} \\
    \sdocHref{Apply}\footnotemark    & \fThree{\fa}{F[A => B]}{\fb} \\
    \sdocHref{FlatMap}\footnotemark  & \fThree{\fa}{(A => F[B])}{\fb} \\
    \sdocHref{CoFlatMap}             & \fThree{\fa}{(F[A] => B)}{\fb} \\
    \sdocHref{Traverse}\footnotemark & \fThree{\fa}{(A => G[B])}{G[F[B]]} \\
    \sdocHref{Foldable}              & \fThree{\fa}{(B, (B, A) => B)}{B} \\
    \sdocHref{SemigroupK}               & \fThree{\fa}{\fa}{\fa} \\
    \sdocHref{Cartesian}             & \fThree{\fa}{\fb}{F[(A, B)]} \\
  \end{tabular}
  \label{tab:normaltab}
\end{table}

% hrm need manual numbering here
\footnotetext[2]{\texttt{Apply} has a (broader) subtype \texttt{Applicative}. See the expanded tables below.}

\footnotetext[3]{\texttt{FlatMap} has a (broader) subtype \texttt{Monad}.}

% hrm need manual numbering here
\footnotetext[4]{\texttt{Traverse} requires that the target type constructor \texttt{G} have an implicit \texttt{Applicative} instance available; that is, an implicit \texttt{Applicative[G]} must be in scope. \\ Informally, traversing a structure maps each value to some effect, which are combined into a single effect that produces a value having the original structure. For example, by transforming every \texttt{A} of a \texttt{List[A]} into a \texttt{Future[B]}, the traversal would return a \texttt{Future[List[B]]}.}

\newpage

\section{Derived Functions}

For each typeclass, its defining function is marked in \textbf{bold} and each derived function listed below it.

\begin{table}[ht]
  \centering
  \fontfamily{ppl}\selectfont
  \begin{tabular}{rrcclll}
    Typeclass & \multicolumn{5}{c}{Signature} & Function \\

    \midrule

    \multirow{6}{*}{\sdocHref{Functor}}
      & \fThree{\multirow{6}{*}{\fa}}{(A => B)}{\fb} & \textbf{map} \\
      & \fThree{}{(A => B)}{F[(A, B)]} & fproduct \\
      & \fThree{}{B}{\fb} & as \\
      & \fThree{}{B}{F[(B, A)]} & tupleLeft \\
      & \fThree{}{B}{F[(A, B)]} & tupleRight \\
      & \fTwo{}{F[Unit]} & void \\[.5cm]

    \multirow{1}{*}{\sdocHref{Contravariant}}
      & \fThree{\fa}{(B => A)}{\fb} & \textbf{contramap} \\[.5cm]

    \multirow{2}{*}{\sdocHref{Apply}\footnotemark}
      & \fThree{\multirow{2}{*}{\fa}}{F[A => B]}{\fb} & \textbf{ap} \\
      & \fThree{}{F[B] => ((A, B) => C)}{F[C]} & \texttt{map2} \\[.5cm]

    \multirow{4}{*}{\sdocHref{Applicative}}
      & \fThree{\multirow{4}{*}{\fa}}{F[A => B]}{\fb} & \textbf{ap} \\
      & \fThree{}{Boolean}{F[Unit]} & unlessA \\
      & \fThree{}{Boolean}{F[Unit]} & whenA \\
      & \fThree{}{Int}{F[List[A]]} & replicateA \\[.5cm]

    \multirow{5}{*}{\sdocHref{FlatMap}}
      & \fThree{\multirow{4}{*}{\fa}}{(A => F[B])}{\fb} & \textbf{flatMap} \\
      & \fThree{}{\fb}{\fb} & followedBy \\
      & \fThree{}{\fb}{\fa} & forEffect \\
      & \fThree{}{(A => \fb)}{F[(A, B)]} & mproduct \\
      & \fThree{F[F[A]]}{}{\fa} & flatten \\[.5cm]

    \multirow{2}{*}{\sdocHref{CoFlatMap}}
      & \fThree{\multirow{2}{*}{\fa}}{(\fa => B)}{\fb} & \textbf{coflatMap} \\
      & \fThree{}{}{F[A[A]]} & coflatten \\[.5cm]

  \end{tabular}
\end{table}

\footnotetext[5]{Both the \texttt{Apply} and \texttt{Applicative} typeclasses implement the \texttt{ap} method; \texttt{Applicative} is a subtype of \texttt{Apply}, with an additional \texttt{pure} method to lift a value into the \texttt{Applicative}.}

\begin{table}[ht]
  \centering
  \fontfamily{ppl}\selectfont
  \begin{tabular}{rrcclll}
    Typeclass & \multicolumn{5}{c}{Signature} & Function \\
    \midrule

    \multirow{4}{*}{\sdocHref{Traverse}}
      & \fThree{\multirow{3}{*}{\fa}}{(A => G[B])}{G[F[B]]} & \textbf{traverse} \\
      & \fThree{}{((A, Int) => B)}{\fb} & mapWithIndex \\
      & \fThree{}{}{F[(A, Int)]} & zipWithIndex \\
      & \fTwo{F[G[A]]}{G[F[A]]} & sequence \\[.5cm]

    \multirow{11}{*}{\sdocHref{Foldable}}
      & \fThree{\multirow{11}{*}{\fa}}{B => ((B, A) => B)}{B} & \textbf{foldLeft} \\
      & \fThree{}{Eval[B] => ((A, Eval[B]) => Eval[B])}{Eval[B]} & \textbf{foldRight} \\
      & \fThree{}{(A => B)}{B} & foldMap\footnotemark \\
      & \fTwo{}{A} & combineAll\footnotemark \\
      & \fThree{}{(A => Boolean)}{Option[A]} & find \\
      & \fThree{}{(A => Boolean)}{Boolean} & exists \\
      & \fThree{}{(A => Boolean)}{Boolean} & forall \\
      & \fTwo{}{List[A]} & toList \\
      & \fTwo{}{Boolean} & isEmpty \\
      & \fTwo{}{Boolean} & nonEmpty \\
      & \fTwo{}{Int} & size \\[.5cm]

    \sdocHref{SemigroupK}
      & \fThree{\fa}{\fa}{\fa} & \textbf{combine} \\[.5cm]

    \sdocHref{Cartesian}
      & \fThree{\fa}{\fb}{F[(A, B)]} & \textbf{product} \\[.5cm]

  \end{tabular}
\end{table}

\footnotetext[6]{If \texttt{B} has a \texttt{Monoid}}
\footnotetext[7]{If \texttt{A} has a \texttt{Monoid}}

\bibliography{sample-handout}
\bibliographystyle{plainnat}

\fancyfoot[C]{\copyright 2017 Adam S. Rosien (\href{mailto:adam@rosien.net}{\texttt{adam@rosien.net}}) \\
This work is licensed under a \href{http://creativecommons.org/licenses/by/4.0/}{Creative Commons Attribution 4.0 International License}. \\
Issues and suggestions welcome at \href{https://github.com/arosien/cats-cheatsheets}{\texttt{https://github.com/arosien/cats-cheatsheets}}}

\end{document}
